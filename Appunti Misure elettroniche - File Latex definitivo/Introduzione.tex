\section*{Introduzione}

 

Appunti ordinati, con approfondimenti passo-passo, del corso di Misure elettroniche per il corso di laurea in Ingegneria Elettronica 
presso l’Università Politecnica delle Marche. \newline
        



Le fonti degli appunti sono le seguenti: 

\begin{itemize}
    
    \item Slide del corso della prof.ssa Susanna Spinsante 
    "Strumentazione Digitale e Misure Elettroniche (SDME)" 
    A.A. 2024/2025 
    

\end{itemize}

La mitica Spinstante è una goat: prof stupenda, riesce a mantenere coinvolgente la materia rimanendo professionale e puntuale nella terminalogia. \newline 

Andate a lezione, se potete, perchè sono super-interessanti. \newline 

È consigliato studiare e superare prima l’esame di analisi matematica 2 e teoria dei segnali per gli indicatori statistici che saranno impiegati nel corso. \newline 

I concetti importanti dei corsi precedenti sono importanti ma non fondamentali: ci serviranno i concetti base. \newline 

Lascerò dei link a video e/o spiegazioni esterne per ulteriori approfondimenti. \newline 

Per qualsiasi domanda, scrivimi a \href{mailto:rossini.stefano.appunti@gmail.com}{rossini.stefano.appunti@gmail.com} \newline

Buono studio e buona lettura \newline

\newpage 





